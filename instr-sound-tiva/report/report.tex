\documentclass[a4paper,9pt,twocolumn]{article}
\usepackage[utf8]{inputenc}
\usepackage[brazilian]{babel}
\usepackage{blindtext}
\usepackage[pdftex]{graphicx}
\usepackage{subfigure}
\usepackage{listings}

\newcommand{\fig}[4][ht!]{
  \begin{figure}[#1]
    {\centering{\includegraphics[#4]{#2}}\par}
    \caption{#3}
    \label{fig:#2}
  \end{figure}
}

% Title
\title{Relatório: Projeto, Implementação e Caracterização de um Instrumento Medidor de Frequência}
\author{Lucas Pires Camargo}

\begin{document}
\maketitle

\begin{abstract}
Este relatório apresenta um instrumento medidor de frequência construído para avaliação na disciplina de Instrumentação. 
São apresentados aspectos gerais da construção do instrumento e depois é apresentao o processo de caracterização, as 
características obtidas do instrumento, e uma discussão dos resultados obtidos. O instrumento operou corretamente e
foi apontado como melhorar sua precisão.
\end{abstract}

%\section{Conceitos Básicos}
%\blindtext

\section{O Instrumento de Medição}

O instrumento foi construído em três partes: o módulo transdutor, uma placa de aquisição de dados, 
e um programa de computador para análise da captura e exibição. A figura~\ref{fig:impl} mostra uma visão geral do sistema.

\fig{impl}{Visão geral do instrumento.}{width=\columnwidth}

O módulo de microfone é uma placa genrérica detectora de som (\emph{clap}) com um microfone. Ela é proveniente de um kit hobbysta e não dispõe 
de datasheet. Ela apresenta apenas uma única saída digital, proveniente de um comparador analógico LM393 e que não é adequada 
para a aplicação desejada. Portanto, ela foi modificada para obtenção do sinal analógico puro. 
Um pino de saída analógica extra foi adicionado soldando-se um conector na saída do estágio de amplificação do microfone. 

A placa de captura foi criada com a plataforma de desenvolvimanto Tiva-C Series TMC123G, fabricado pela Texas Instruments.
O programa é simples: faz a leitura do módulo conversor A/D em uma frequência específica e envia os dados ao computador através da porta serial.
Os requisitos de tempo são atendidos utilizando-se de interrupções para o disparo do timer. Há a garantia de que o tempo de conversão máximo do módulo A/D
é sempre uma ordem de grandeza menor do que o inverso da frequência de captura (máximo $1\mu{}s$). A velocidade do módulo serial é de 460800~baud, 
na configuração 8~bits de dados, nenhum bit de paridade, e 1~bit de parada. A velocidade resultante é de 46080~kBps, adequada para a transmissão 
de amostras de 8~bits a 22050KHz.

Por fim foi feito o programa de computador que faz a análise espectral do sinal e exibe a componente de frequência principal. 
O núcleo do programa foi escrito em na liguagem de programação C++ e a interface com o usuário na linguagem declarativa QML.
Foi utilizada a biblioteca \texttt{fftw} para executar a transformada de fourier dos buffers de amostras. A entrada é analisada
à taxa de 8 buffers pro segundo. O programa possibilita a normalização da onda sonora para exibição, e a aplicação do Filtro Janela 
Hann no sinal de entrada, ativado pro padrão. A Figura~\ref{fig:app1} mostra a aplicação executando no desktop. A Figura~\ref{fig:setup} 
mostra a configuração de testes completa.

\fig{app1}{Aplicação medidora de frequência.}{width=0.8\columnwidth}

\fig{setup}{Configuração de teste.}{width=\columnwidth}

\section{Caracterização}
Pode-se avaliar a faixa de valores de entrada e saida teóricos através do teorema da amostragem de Nyquist–Shannon. Frequências superiores à 
metade da frequência de amostragem causam o fenômeno de \emph{aliasing}. Portanto a máxima frequência que pode ser verificada é 22050~kHz/2 = 11025~kHz.

O instrumento analisa a cada 0,125~s um \emph{buffer} de $N=2756$~amostras. A transformada de fourier discreta utilizada na análise apresenta na 
$N_f = N/2 = 1378$ níveis de frequência. Esse níveis devem ser escalonados linermente para a faixa $[0, f/2]~Hz$ de valores de saída do instrumento, onde f é a frequência de amostragem
do sinal de entrada. Portanto, a resolução do instrumento é dada por $f_2/N_f = 8$~Hz. 

Foram feitas várias medições de teste, apresentadas na Tabela 1. Estabeleceu-se que o instrumento apresenta zonas mortas nas 
faixas $[0, 100[~Hz$ e $]10000, 11000]~Hz$. Essas zonas mortas podem ser atribuídas à baixa resposta do sensor nessas faixas de frequência.
Das medições realizadas foi derivada a curva de calibração apresentada na Figura~\ref{fig:data}. 

\fig{data}{Curva de calibração do sensor.}{width=\columnwidth}

O medidor pôde ser analisado estaticamente. Não foram feitas análises dinâmicas. Também não foi observada nenhuma histerese. Pode se 
atribuir essas características desejáveis ao fato de o programa medidor processar seções discretas do sinal de maneira independente umas das outras.

Uma possível aplicação do instrumento seria na análise de ruído 

\section{Discussão}

Foi observado que, à exceção das zonas mortas, o erro das medições não ultrapassou o limite de resolução do instrumento. Portanto, o 
principal limitante da performance do instrumento são os erros de quantização. Pode ser demonstrado que na implementação atual a resolução em $Hz$ sempre
será igual à taxa de atualização do instrumento. Portanto a resolução pode ser melhorada diminuindo-se a taxa de atualização, 
sacrificando responsividade.

O instrumento poderia ser melhorado tornando a taxa de atualização configurável, e explicitando-se qual a resolução da medição atual. Um microfone 
com uma resposta de frequência mais uniforme poderia diminuir as zonas mortas do instrumento.


\section{Conclusão}

Este relatório apresentou a implementação de um instrumento medidor de frequência e a análise de suas características. Verificou-se que o
instrumento fncionou como o proposto. Foi feita uma análise de suas características principais e foram feitas sugestões para sua melhoria
e desenvolvimentos futuros.


\begin{table*}
\begin{center}
 

  \begin{tabular}{lrr}
    Entrada & Saída & Erro \\ \hline
    50  &  0  &  50 \\
    75  &  0  &  75 \\
    100 & 96  &  4 \\
    150 & 152 &  2 \\
    200 & 200 &  0 \\
    300 & 296 &  4 \\
    400	& 400	& 0     \\
500	& 504	& 4     \\
600	& 600	& 0     \\
700	& 704	& 4     \\
800	& 800	& 0     \\
900	& 904	& 4     \\
1000	& 1000	& 0     \\
1200	& 1200	& 0     \\
1400	& 1400	& 0     \\
1600	& 1600	& 0     \\
1800	& 1800	& 0     \\
2000	& 2000	& 0     \\
2500	& 2504	& 4     \\
3000	& 3000	& 0     \\
3500	& 3504	& 4     \\
4000	& 4000	& 0     \\
4500	& 4504	& 4     \\
5000	& 5000	& 0     \\
5500	& 5504	& 4     \\
6000	& 6000	& 0     \\
6500	& 6504	& 4     \\
7000	& 7000	& 0     \\
7500	& 7504	& 4     \\
8000	& 8000	& 0     \\
8500	& 8504	& 4     \\
9000	& 9000	& 0     \\
9500	& 9504	& 4     \\
10000	& 10000	& 0     \\
10500	& 0	& 10500 \\
11000	& 0	& 11000 \\

  \end{tabular}
  \caption{Resultados das medidas de teste}
  \label{tab:ch}

  \end{center}
\end{table*}




\end{document}          
