\documentclass[a4paper,9pt,twocolumn]{article}
\usepackage[utf8]{inputenc}
\usepackage[brazilian]{babel}
\usepackage{blindtext}
\usepackage[pdftex]{graphicx}
\usepackage{subfigure}
\usepackage{listings}

\newcommand{\fig}[4][ht!]{
  \begin{figure}[#1]
    {\centering{\includegraphics[#4]{#2}}\par}
    \caption{#3}
    \label{fig:#2}
  \end{figure}
}

% Title
\title{Relatório: Projeto, Implementação e Caracterização de um Instrumento Medidor de Velocidade Angular}
\author{Lucas Pires Camargo}

\begin{document}
\maketitle

\begin{abstract}
Este relatório apresenta um instrumento medidor de velocidade angular construído para avaliação na disciplina de Instrumentação. 
São apresentados aspectos gerais da construção do instrumento e depois é apresentao o processo de caracterização, as 
características obtidas do instrumento, e uma discussão dos resultados obtidos. O instrumento operou corretamente e
foi apontado como melhorar sua precisão.
\end{abstract}

%\section{Conceitos Básicos}
%\blindtext

\section{O Instrumento de Medição}

O instrumento foi construído em três partes: o módulo transdutor, uma placa de aquisição de dados, 
e um programa de computador para análise da captura e exibição. A Figura~\ref{fig:impl} mostra uma visão geral do sistema.

\fig{impl}{Visão geral do instrumento.}{width=\columnwidth}

O módulo transdutor é um \emph{encoder} de quadratura simples com 30 pulsos (15 ciclos) por rotação. Este é proveniente de um kit hobbysta e não dispõe 
de datasheet. Um \emph{encoder} de quadratura é composto por dois contatos que deslizam sobre a região dentada de uma engrenagem, com um deslocamento entre si. A rotação da engrenagem sob os contatos gera duas ondas quadradas com uma defasagem. A Figura~\ref{fig:quadrature} mostra os dois sinas gerados pelos contatos. Quando o sentido de rotação se inverte, a defasagem também é invertida. É possível determinar um deslocamento angular do eixo do \emph{encoder} contando os pulsos dos sinais de entrada. Questões importantes que surgem da utilização de \emph{encoders} de quadratura: a natureza mecânica dos contatos cria a necessidade do condicionamento dos sinais por uma técnica conhecida como \emph{debouncing}, que busca eliminar o efeito de múltiplos acionamentos causado pelo ressalto dos contatos.


\fig{quadrature}{Sinais de saída de um \emph{encoder} de quadratura.}{width=\columnwidth}

A placa de captura foi criada com a plataforma de desenvolvimento Tiva-C Series TMC123G, fabricada pela Texas Instruments.
O programa lê continuamente os sinais provenientes do \textit{encoder} e determina a direção de deslocamento do eixo quando ocorre um pulso. O debouncing é feito de duas maneiras. A primeira é evitando-se que o sinal de entrada do algoritmo possa mudar muito rapidamente, o que elimina pulsos extremamente curtos de serem avaliados. A segunda maneira trata da validação das mudanças de estado dos sinais de entrada. Comparando-se o estado atual com o anterior, pode ser determinada a direção da rotação e se a mesma deve ser entendida como um movimento de avanço ou ignorado. 

Quando um pulso é reconhecido, ele é enviado ao computador na forma de um único caractere na porta serial. A velocidade do módulo serial é de 115200~baud, 
na configuração 8~bits de dados, nenhum bit de paridade, e 1~bit de parada. A velocidade resultante é de 11520~kBps, ou pulsos por segundo. Com o \textit{encoder} sendo utilizado com o instrumento fornecendo 30 pulsos por ciclo, isso se traduz em uma velocidade máxima teórica de 384 rotações por segundo, ou seja, 23040~rpm.

Por fim foi feito o programa de computador que faz a contagem dos pulsos e exibe a velocidade de rotação estimada. O núcleo do programa foi escrito em na liguagem de programação C++ e a interface com o usuário na linguagem declarativa QML. O programa possui duas configurações: a quantidade de pulsos emitidos pelo \textit{encoder} por rotação, e a duração do intervalo de contagem de pulsos. O programa possibilita a vizualização dos pulsos em tempo real, através de um mostrador circular simples. Também avisa se houverem pulsos em direções discordantes em um mesmo intervalo de medição, o que indica problemas de aquisição para uma carga que gira consistentemente em uma direção. A Figura~\ref{fig:app1} mostra a aplicação executando no desktop. 

\fig{app1}{Aplicação medidora de velocidade angular.}{width=0.8\columnwidth}


\section{Caracterização}
Para avaliação do instrumento, foi necessário acoplar o \textit{encoder} a uma carga e verificar seu correto funcionamento. Um antigo servomotor escovado de íma permanente E760 da Reliance Electric foi utilizado para essa finalidade. Para realizar o acoplamento, foi necessário criar um suporte para o \textit{encoder} no motor e um acoplador fêmea-fêmea em um torno. O projeto dos mesmos foi realizado com o auxílio do software Solidworks. A Figura~\ref{fig:solidworks} mostra o projeto do acoplamento em vista explodida. Este foi usinado e montado no Laboratório de Fabricação da UFSC Joinville. A Figura~\ref{fig:setup} mostra o instrumento acoplado ao motor.

\fig{solidworks}{Projeto do acoplamento no Solidworks.}{width=\columnwidth}

\fig{setup}{Instrumento acoplado ao motor.}{width=\columnwidth}

Na configuração padrão, o instrumento conta pulsos em intervalos de 400~ms. Dados $N$= número de pulsos contabilizados, $E$= número de pulsos por revolução do \textit{encoder}, e $T$ = tamanho do intervalo de medição em segundos,  a medição em rotações por minuto é dada por $\omega_{rpm} = (60 * N)/(E * T)$. Como a medição de saída é inversamente proporcional à contagem de pulsos do \textit{encoder}, a suscetibikidade do instrumento au jittering não é constante. Quanto maior a velocidade de rotação (quantidade de pulsos por revolução do \textit{encoder}) ou janela de amostragem, melhor é a medição da velocidade, em proporção à sua magnitude.

A utilização de CPU média na execução dos testes foi cerca de 5\%. Para obtenção de um valor de referência para a caracterização do instrumento, um sensor indutivo contador de pulsos foi instalado em proximidade do eixo do \textit{encoder} e a frequência de rotação do motor foi contabilizada por um osciloscópio KEYSIGHT InfiniVision DSO-X-3014A. Os testes foram executados no sistema operacional Linux 4.2.0-18 em um computador de mesa pessoal convencional. 

Foram feitas várias medições de teste, apresentadas na Tabela 1. De maneira geral, a resposta do instrumento foi muito próxima à saída esperada, fornecendo uma curva de calibração fortemente linear. Pôde ser observado que a amostra de maior magnitude apresentou maior erro, o que pode indicar o início de perdas da contagem de pulsos. Uma maior velocidade não pôde ser amostrada devido a limitações do motor sendo utilizado.
Das medições realizadas foi derivada a curva de calibração apresentada na Figura~\ref{fig:data2}. As linhas do gráfico se sobrepõem, portanto apenas uma linha é visível com clareza. No gráfico, "medidor rpm" denota a medição do instrumento apra um dado valor de entrada, e "função calibração inv" denota a curva de calibração obtida por uma regressão linear.

\fig{data2}{Dados e curva de calibração.}{width=\columnwidth}

O medidor pôde ser analisado estaticamente. Não foram feitas análises dinâmicas. Também não foi observada nenhuma histerese. Pode se atribuir essas características desejáveis ao fato de o programa medidor processar seções discretas do sinal de maneira independente umas das outras.

Uma possível aplicação do instrumento é o controle de velocidade para aplicações de servomotores em laboratório e a prototipagem de novos equipamentos. O instrumento foi implementado em uma plataforma que dispõe de poder computacional disponível e periféricos adequados (geradores de ondas PWM) para a implementação de controle de velocidade de maneira independente, sem auxílio de um computador. 

\section{Discussão}

Durante o processo de amostragem, a medição oscilava em torno de um valor médio. Pôde ser verificado que os intervalos de medição fixos e leituras da porta serial causavam \textit{jittering}, que foi o principal limitante da performance do instrumento. A resolução pode ser melhorada diminuindo-se a taxa de atualização, sacrificando responsividade, ou acrescentando timestamps às amostras do programa, o que deve reduzir o jittering consideravelmente.

O instrumento poderia ser melhorado tornando a taxa de atualização configurável, e explicitando-se qual a resolução da medição atual. Um \textit{encoder} menos ruidoso poderia diminuir a chance de ocorrerem perdas na contagem de pulsos.


\section{Conclusão}

Este relatório apresentou a implementação de um instrumento medidor de frequência e a análise de suas características. Verificou-se que o
instrumento fncionou como o proposto. Foi feita uma análise de suas características principais e foram feitas sugestões para sua melhoria
e desenvolvimentos futuros.


\begin{table}
\begin{center}
 

  \begin{tabular}{lrr}
    Entrada & Saída & Erro \\ \hline
    63  &  65  &  2 \\
    111,3  &  111  &  -0,3 \\
    283,8 & 285  &  1,2 \\
    477 & 475 &  -2 \\
    837,6 & 837 &  -0,6 \\
    1080 & 1082 &  2 \\
    1380 & 1380 & 0     \\
1609,8	& 1610	& 0,2     \\
1931,4	& 1928	& -3,4     
  \end{tabular}
  \caption{Resultados das medidas de teste}
  \label{tab:ch}

  \end{center}
\end{table}




\end{document}          
